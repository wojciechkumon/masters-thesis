\documentclass[11pt]{aghdpl}
\usepackage[english,polish]{babel}

\usepackage{polski}

\usepackage[utf8]{inputenc}


\usepackage{mathtools}
\usepackage{amsfonts}
\usepackage{amsmath}
\usepackage{amsthm}
\usepackage{tikz}
\usepackage{algorithm}
\usepackage{algorithmic}

\usepackage[
style=numeric,
sorting=none,
%
% Zastosuj styl wpisu bibliograficznego właściwy językowi publikacji.
language=autobib,
autolang=other,
% Zapisuj datę dostępu do strony WWW w formacie RRRR-MM-DD.
urldate=iso8601,
% Nie dodawaj numerów stron, na których występuje cytowanie.
backref=false,
% Podawaj ISBN.
isbn=true,
% Nie podawaj URL-i, o ile nie jest to konieczne.
url=false,
%
% Ustawienia związane z polskimi normami dla bibliografii.
maxbibnames=3,
% Jeżeli używamy BibTeXa:
backend=bibtex
]{biblatex}

\usepackage{csquotes}
% Ponieważ `csquotes` nie posiada polskiego stylu, można skorzystać z mocno zbliżonego stylu chorwackiego.
\DeclareQuoteAlias{croatian}{polish}

\addbibresource{bibliografia.bib}

% Nie wyświetlaj wybranych pól.
%\AtEveryBibitem{\clearfield{note}}


\usepackage{courier}

\usepackage{listings}
\lstloadlanguages{TeX}

\lstset{
	literate={ą}{{\k{a}}}1
           {ć}{{\'c}}1
           {ę}{{\k{e}}}1
           {ó}{{\'o}}1
           {ń}{{\'n}}1
           {ł}{{\l{}}}1
           {ś}{{\'s}}1
           {ź}{{\'z}}1
           {ż}{{\.z}}1
           {Ą}{{\k{A}}}1
           {Ć}{{\'C}}1
           {Ę}{{\k{E}}}1
           {Ó}{{\'O}}1
           {Ń}{{\'N}}1
           {Ł}{{\L{}}}1
           {Ś}{{\'S}}1
           {Ź}{{\'Z}}1
           {Ż}{{\.Z}}1,
	basicstyle=\footnotesize\ttfamily,
}

\AtBeginDocument{
	\renewcommand{\tablename}{Tabela}
	\renewcommand{\figurename}{Rys.}
}


\usepackage{array}
\usepackage{tabularx}
\usepackage{multirow}
\usepackage{booktabs}
\usepackage{makecell}
\usepackage[flushleft]{threeparttable}

\newcolumntype{C}[1]{>{\hsize=#1\hsize\centering\arraybackslash}X}


%---------------------------------------------------------------------------

\author{Wojciech Kumoń}
\shortauthor{W. Kumoń}

%\titlePL{Przygotowanie bardzo długiej i pasjonującej pracy dyplomowej w~systemie~\LaTeX}
%\titleEN{Preparation of a very long and fascinating bachelor or master thesis in \LaTeX}

\titlePL{Implementacja algorytmu weryfikacji modelowej własności LTL w środowisku rozproszonym}
\titleEN{Implementation of the distributed LTL model checking algorithm}


\shorttitlePL{Implementacja algorytmu weryfikacji modelowej własności LTL w środowisku rozproszonym}
\shorttitleEN{Implementation of the distributed LTL model checking algorithm}

\thesistype{Praca dyplomowa magisterska}

\supervisor{prof. dr hab. Marcin Szpyrka}
\degreeprogramme{Informatyka}
\date{2019}
\department{Katedra Informatyki Stosowanej}
\faculty{Wydział Elektrotechniki, Automatyki,\protect\\[-1mm] Informatyki i Inżynierii Biomedycznej}

\acknowledgements{Serdecznie dziękuję \dots tu ciąg dalszych podziękowań np. dla promotora, żony, sąsiada itp.}


\setlength{\cftsecnumwidth}{10mm}

\setcounter{secnumdepth}{4}
\brokenpenalty=10000\relax

\begin{document}

\titlepages

% Ponowne zdefiniowanie stylu `plain`, aby usunąć numer strony z pierwszej strony spisu treści i poszczególnych rozdziałów.
\fancypagestyle{plain}
{
	% Usuń nagłówek i stopkę
	\fancyhf{}
	% Usuń linie.
	\renewcommand{\headrulewidth}{0pt}
	\renewcommand{\footrulewidth}{0pt}
}

\setcounter{tocdepth}{2}
\tableofcontents
\clearpage

\chapter{Wstęp}

Weryfikacja modelowa to dziedzina umożliwiająca sprawdzenie systemu pod kątem specyfikacji.
Operacja taka jest zazwyczaj bardzo wymagająca pod kątem obliczeniowym, co skutkuje długimi czasami wykonania. Przeciwdziałać temu można na kilka sposobów, np. stosując uproszczenie modelu, czy wykorzystanie wydajniejszego procesora. Kolejna możliwość to stworzenie skalowanego systemu rozproszonego i właśnie ta metoda zostanie rozważona.

Samą specyfikację wyrazić można na wiele sposobów. W pracy wykorzystana zostanie logika LTL (ang. \textit{linear-time temporal logic}).


\section{Cel pracy}

Celem pracy jest implementacja rozproszonego algorytmu weryfikacji modelowej w oparciu o własności logiczne czasu liniowego - LTL dla języka Alvis z wykorzystaniem frameworka Spring.


\section{Struktura pracy}

W następnym rozdziale pracy omówione zostanie zagadnienie weryfikacji modelowej w kontekście tematu pracy.
Rozdział trzeci zawiera opis stworzonego rozwiązania.
W czwartym rozdziale znajduje się prezentacja uzyskanych wyników.
Piąty rozdział zawiera podsumowanie.

\chapter{Weryfikacja modelowa}

Systemy tworzone przez ludzi są coraz bardziej złożone oraz odgrywają coraz większą rolę w życiu każdego z nas.
Błędy w oprogramowaniu skutkują stratami finansowymi, wizerunkowymi, opóźnieniami, a także utratą zdrowia i życia ludzi. Dowodzą temu następujące przykłady: nieudany start Ariane-5 (04.06.1996), błąd w procesorach Pentium II Intela, czy źle działająca maszyna do radioterapii spowodowała śmierć sześciu pacjentów w latach 1985-1987.


\section{Weryfikacja systemu}

Weryfikacja systemu ma na celu ustalelnie, czy projekt posiada oczekiwane właściwości. Mogą one być dość podstawowe, np. nigdy nie dojdzie do zakleszczenia lub związane z domeną, np. nie można wypłacić więcej pieniędzy niż jest na koncie. Specyfikacja dostarcza informacji jak system może oraz jak nie może się zachowywać. Oprogramowanie uważa się za poprawne, jeśli spełnia wszystkie właściwości. Schemat weryfikacji został przedstawiony na rys. \ref{fig:system_verification_scheme}.

\begin{figure}[h]
    \centering
    \includegraphics[height=7.5cm,keepaspectratio]{img/system_verification_schematic_view.png}
    \caption{Schemat tworzenia systemu wraz z jego weryfikacją (źródło~\cite{Bai08}).}
    \label{fig:system_verification_scheme}
\end{figure}

Podstawową formą radzenia sobie z tym problemem jest testowanie oprogramowania (testy jednostkowe, integracyjne, systemowe itp.). Polega to na uruchamianiu kodu dla różnych ścieżek wykonania, po czym waliduje się ich wyjścia z oczekiwanymi. Niestety, przetestowanie wszystkiego okazuje się praktycznie niemożliwe, zwykle sprawdzane są jedynie warunki brzegowe (stanowi to mały podzbiór wszystkich kombinacji). 


\section{Weryfikacja modelowa}

Podczas tworzenia skomplikowanych systemów, kładzie się coraz większy nacisk na testowanie poprawności oprogramowania. Metody formalne mają duży potencjał na tym polu. Ich wczesna integracja (podczas procesu projektowania) dostarcza efektywnych technik weryfikacji.
Intuicyjnie, metody formalne można rozważać jako matematykę stosowaną dla modelowania i analizy systemów informatycznych. Zapewniają one poprawność z matematyczną dokładnością.

Techniki weryfikacji bazujące na modelu opisują zachowanie systemu deterministycznie i kompletnie. Samo tworzenie pełnego modelu może wykryć luki lub niespójności.
Po jego stworzeniu, wraz z otoczającymi algorytmami, możliwe jest eksplorowanie stanów systemu.
Dzieje się to w podejściu "brute-force" - przejrzane zostają wszystkie możliwe scenariusze.
W ten sposób udowadnia się spełnialność właściwości.




\begin{figure}[h]
    \centering
    \includegraphics[height=8.8cm,keepaspectratio]{img/model_checking_approach_schematic_view.png}
    \caption{Schemat podejscia weryfikacji modelowej (źródło~\cite{Bai08}).}
    \label{fig:model_checking_scheme}
\end{figure}


\section{Logika LTL}

TODO

weryfikacja modelowa, przestrzen stanow, LTL, automaty, automaty buchiego
weryfikacja hardware i software
alogrytmy wykorzystywane w przeszukiwaniu i weryfikacji
Alvis, co to, po co

\cite{Bar12} \cite{Jac05}
przejrzec publikacje profesora




w czesci implementacji:
opis weryfikowanego systemu


A time-optimal on-the-fly parallel algorithm for model checking of weak LTL properties, in: Formal Methods and Software Engineering

 

\chapter{Architektura systemu}

Założeniem projektu było stworzenie algorytmu weryfikacji modelowej w oparciu o własności LTL dla języka Alvis w środowisku rozproszonym.

Alvis to język formalny, którego celem jest dostarczenie elastyczności w modelowaniu systemów współbieżnych i czasu rzeczywistego wraz z możliwością weryfikacji opartej o metody formalne.
Stanowi połączenie zalet języków wysokiego poziomu z graficznym budowaniem zależności między podsystemami (nazywanymi agentami).
% TODO jak dużo o Alvisie takiego wstępu?


\section{System rozproszony}

Architektura systemu została zaplanowana tak, aby wydzielić funkcjonalności do odseparowanych aplikacji.
Całość składa się z 3 programów oraz bazy danych.
Jej schemat ogólny został przedstawiony na rys. \ref{fig:system_overview}.

\begin{figure}[h]
    \centering
    \includegraphics[width=\linewidth,keepaspectratio]{img/system_overview.png}
    \caption{Schemat ogólny systemu.}
    \label{fig:system_overview}
\end{figure}

Pierwszym elementem jest \textit{Graph Provider} (\textbf{GP}).
Spełnia on dwa zadania.
Pierwsze z nich to dostarczanie wszystkich osiągalnych tranzycji dla danego stanu.
Drugie umożliwia otrzymanie stanów dla zadanej tranzycji, które można odwiedzić z wybranego stanu źródłowego.
Serwis ten enkapsuluje działanie całej domeny.
Jako jedyny dostarcza dane, dzięki którym da się zbudować całą przestrzeń stanów.
GP spełnia prosty interfejs przedstawiony w listingu \ref{lst:graph_provider_interface}.

\begin{lstlisting}[caption={Interfejs implementowany przez GP.},captionpos=b,label={lst:graph_provider_interface}]

  public interface GraphProvider {
    Collection<Transition> allTransitionFromState(SystemState from);

    Collection<SystemState> allReachableStates(SystemState from,
                                               Transition through);
  }
\end{lstlisting}

\textit{LTL Verifier} (\textbf{LV}) to główna serwerowa aplikacja zajmująca się weryfikacją modelową i zawiera główną część tego algorytmu.
Konwertuje ona także formuły LTL do automatów Büchiego.
Napisana została w języku Java z wykorzystaniem frameworka Spring.
Interfejs implementowany przez LV opisuje listing \ref{lst:ltl_verifier_interface}.

\begin{lstlisting}[caption={Interfejs implementowany przez LV.},captionpos=b,label={lst:ltl_verifier_interface}]

  public interface LtlVerifier {
    VerificationId createVerificationJob(VerificationInit verificationInit);

    VerificationResult newStates(NewStates newStates,
                                 VerificationId id);

    VerificationResult finish(VerificationId id);
  }
\end{lstlisting}

\textit{Graph Builder} (\textbf{GB}) spełnia rolę serca systemu.
To centrum sterowania, które inicjuje zapytania do pozostałych aplikacji.
Zarządza eksploracją kolejnych stanów, a także wysyła je do weryfikacji.
Zarówno GB jak i LV komunikują się z bazą danych (Cassandra), aby czytać i zapisywać stany czy tranzycje.
Komunikacja między GB a GP zachodzi za pomocą Apache Thrift, wykorzystując binarny protokół przesyłając dane siecią.
Zapytania GB do LV realizowane są poprzez protokół HTTP w formacie JSON.

% TODO dopisać o celach i możliwościach architektury, może diagram z wyszczególnionymi protokołami komunikacji

\chapter{Algorytm weryfikacji} \label{chap:model_checking_algorithm}

Zastosowany algorytm weryfikacji modelowej opiera się głównie na \textit{on-the-fly OWCTY} \cite{Bar12}, który został dostosowany do wykorzystania w środowisku rozproszonym.

\begin{algorithm}
\caption{$ detectAcceptingCycle() $}
\label{alg:detectAcceptingCycle}
\begin{algorithmic}[1]
\REQUIRE $ G = (V,E,ACC) $

\STATE $ initialStates \leftarrow getInitialStates() $
\STATE $ approximationSet \leftarrow initialise(initialStates) $
\STATE $ oldSize \leftarrow \infty $
\WHILE{$ (approximationSet.size \neq oldSize)\ \mathbf{and}\ (approximationSet.size > 0) $}
  \STATE $ oldSize \leftarrow approximationSet.size $
  \STATE $ eliminateNoAccepting(approximationSet) $
  \STATE $ eliminateNoPredecessors(approximationSet) $
\ENDWHILE
\RETURN $ approximationSet.size > 0 $
\end{algorithmic}
\end{algorithm}

Algorytm OWCTY wykorzystuje sortowanie topologiczne dla detekcji cykli.
Zapewnia liniową złożoność czasową przy jednoczesnym uniknięciu przeszukiwania w głąb, co umożliwia wykonanie równoległe.
Niestety procedura sortowania topologicznego nie może bezpośrednio wykryć cykli akceptujących.
Zamiast tego wykorzystuje się eliminację cykli nieakceptujących.
Obliczany jest zbiór stanów poprzedzanych przez cykl akceptujący (\textit{approximationSet}).
Jeśli po zakończeniu algorytmu zbiór ten jest pusty, oznacza  to, że nie istnieje cykl akceptujący.
Sam zbiór wylicza się w kilku fazach.
Pierwsza z nich, czyli \textit{initialise()} (algorytm \ref{alg:initialise}) eksploruje pełną przestrzeń stanów systemu oraz przygotowuje niezbędne dane dla kolejnych faz.
Kolejne dwie usuwają ze zbioru stany, które nie mogą być częścią cyklu akceptującego.

Jedna z nich to \textit{eliminateNoAccepting()} (algorytm \ref{alg:eliminateNoAccepting}).
Zaczyna się od pozostawienia w zbiorze jedynie stanów akceptujących (linie 3-10).
Następnie obliczane są wartości liczby poprzedników dla każdego wierzchołka.
To część procedury osiągalności, która poszerza cały zbiór, a ten może zawierać już nie tylko stany akceptujące (linie 11-22).

Ostatnia faza OWCTY to \textit{eliminateNoPredecessors()} (algorytm \ref{alg:eliminateNoPredecessors}).
Bazuje ona na sortowaniu topologicznym.
Wykorzystuje liczbę poprzedników wyliczonych w \textit{eliminateNoAccepting()}, aby iteracyjnie usuwać wierzchołki ze zbioru, kiedy ich stopień (w podgrafie tworzonym przez stany pozostałe w zbiorze) równy jest 0.
Kiedy wierzchołek znika ze zbioru, należy zmniejszyć stopień jego następników o 1.
Brak takich następników oznacza zakończenie fazy.
\textit{EliminateNoAccepting()} oraz \textit{eliminateNoPredecessors()} wykonywane są w pętli, dopóki zachodzą jakieś zmiany.

\noindent
Opis użytych zmiennych/procedur w algorytmach \ref{alg:detectAcceptingCycle} i \ref{alg:initialise}:
\begin{itemize}
\item $ G = (V,E,ACC) $ -- wejściowy graf składający się ze zbiorów wierzchołków, krawędzi i stanów akceptujących
\item \textit{getInitialStates()} -- procedura zwracająca zbiór stanów początkowych dla grafu $G$
\item \textit{isAccepting(x)} -- zwraca prawdę, jeśli stan $x$ jest akceptujący, fałsz w przeciwnym wypadku
\item \textit{acceptingCycleFound()} -- służy do wcześniejszego zakończenia algorytmu (kiedy cykl akceptujący został wykryty bez eksploracji całej przestrzeni stanów)
\end{itemize}

\begin{algorithm}
\caption{$ initialise(initialStates) $}
\label{alg:initialise}
\begin{algorithmic}[1]
\REQUIRE $ initialStates $

\STATE $ approximationSet \leftarrow initialStates $
\STATE $ q \leftarrow new\ Queue() $
\STATE $ q.pushBack(initialStates) $
\WHILE{$ q.isNotEmpty() $}
  \STATE $ s \leftarrow q.popFront() $
  \FORALL{$ t \in getSuccessors(s) $}
    \IF{$ t \notin approximationSet $}
      \STATE $ approximationSet.add(t) $
      \STATE $ q.pushBack(t) $
    \ENDIF
    \IF{$ isAccepting(t) $}
      \IF{$ (t == s)\ \mathbf{or}\ (approximationSet.getMap(s) == t) $}
        \STATE $ acceptingCycleFound() $
        \RETURN
      \ENDIF
      \STATE $ approximationSet.setMap(t, max(t, approximationSet.getMap(s))) $
    \ELSE
      \STATE $ approximationSet.setMap(t, approximationSet.getMap(s)) $
    \ENDIF
  \ENDFOR
\ENDWHILE
\RETURN $ approximationSet $
\end{algorithmic}
\end{algorithm}


\section{Heurystyka} \label{heuristic}

To zaaplikowanie heurystyki w fazie inicjowania (funkcja \textit{initialise()} w algorytmie \ref{alg:initialise}) modyfikuje oryginalny OWCTY.
Jedyna różnica to linie 11-19, które wykorzystują pomysł z algorytmu MAP.
Propaguje się jednego akceptującego poprzednika przez wszystkie nowo odkryte krawędzie.
Jeśli stan akceptujący zostanie przekazany do samego siebie, oznacza to wykrycie cyklu akceptującego, a obliczenia zostają przerwane (linia 13).
Zgodnie z działaniem algorytmu MAP, na stan akceptujący do rozpropagowania wybiera się ten maksymalny spośród stanów akceptujących odwiedzonych na ścieżce ze stanu początkowego do obecnego.

\begin{algorithm}
\caption{$ eliminateNoAccepting(approximationSet) $}
\label{alg:eliminateNoAccepting}
\begin{algorithmic}[1]
\REQUIRE $ approximationSet $

\STATE $ tmpApproximationSet \leftarrow \emptyset $
\STATE $ q \leftarrow new\ Queue() $
\FORALL{$ s \in approximationSet $}
  \IF{$ isAccepting(s) $}
    \STATE $ q.pushBack(s) $
    \STATE $ tmpApproximationSet.add(s) $
    \STATE $ tmpApproximationSet.setPredecessorCount(s,0) $
  \ENDIF
\ENDFOR
\STATE $ approximationSet \leftarrow tmpApproximationSet $
\WHILE{$ q.isNotEmpty() $}
  \STATE $ s \leftarrow q.popFront() $
  \FORALL{$ t \in getSuccessors(s) $}
    \IF{$ t \in approximationSet $}
      \STATE $ approximationSet.incrementPredecessorCount(t) $
    \ELSE
      \STATE $ q.pushBack(t) $
      \STATE $ approximationSet.add(t) $
      \STATE $ approximationSet.setPredecessorCount(t,0) $
    \ENDIF
  \ENDFOR
\ENDWHILE
\end{algorithmic}
\end{algorithm}

Faza inicjacji algorytmu OWCTY wymaga eksploracji całej przestrzeni stanów, więc została użyta do wykonania detekcji cykli, wykorzystując propagację maksymalnego akceptującego stanu.
W przeciwieństwie do algorytmu MAP brakuje tu repropagacji, aby złożoność obliczeniowa pozostała i liniowa i była proporcjonalna do rozmiaru grafu.
Skutek tego ograniczenia to brak wykrywalności wszystkich cykli (stąd klasyfikacja tej fazy jako heurystyka).

\noindent
Wyróżnić można dwa podstawowe przypadki, kiedy metoda ta będzie nieskuteczna w wykryciu istniejącego cyklu akceptującego:
\begin{enumerate}
  \item Maksymalny akceptujący poprzednik nie leży wewnątrz cyklu.
  \item Wartość maksymalnego akceptującego poprzednika nie wróci do źródła, mimo że cykl istnieje.
\end{enumerate}

\begin{algorithm}
\caption{$ eliminateNoPredecessors(approximationSet) $}
\label{alg:eliminateNoPredecessors}
\begin{algorithmic}[1]
\REQUIRE $ approximationSet $

\STATE $ q \leftarrow new\ Queue() $
\FORALL{$ s \in approximationSet $}
  \IF{$ approximationSet.getPredecessorCount(s) == 0 $}
    \STATE $ q.pushBack(s) $
  \ENDIF
\ENDFOR
\WHILE{$ q.isNotEmpty() $}
  \STATE $ s \leftarrow q.popFront() $
  \STATE $ approximationSet.remove(s) $
  \FORALL{$ t \in getSuccessors(s) $}
    \STATE $ approximationSet.decrementPredecessorCount(t) $
    \IF{$ approximationSet.getPredecessorCount(t) == 0 $}
      \STATE $ q.pushBack(t) $
    \ENDIF
  \ENDFOR
\ENDWHILE
\end{algorithmic}
\end{algorithm}

Pierwszy przypadek obsługiwany jest w oryginalnym algorytmie MAP poprzez iteracyjne usuwanie stanów akceptujących, co wymaga dodatkowej liczby przejść o liniowej złożoności.
Drugi przypadek obsługuje repropagacja, która również nie mogła zostać zawarta ze względu na podniesienie złożoności obliczeniowej.

W sytuacji, gdy żadne z powyższych nie zachodzi, akceptujący cykl zostanie wykryty.
Przykłady z rys. \ref{fig:alg_heuristic_examples}:
\begin{enumerate}[label=(\alph*)]
\item Przypadek trywialny (jeden stan akceptujący). Wartość $B$ zostanie rozpropagowana do $C$ i $D$. W wyniku tego $B$ powróci do węzła źródłowego, więc cykl zostanie wykryty.
\item W tej sytuacji także nastąpi wykrycie cyklu, jednak po drodze występuje kilka stanów akceptujących. Ponieważ $ B > C \land B > D $, $B$ zostanie przesłane dalej.
\item Przypadek podobny do poprzedniego, lecz krawędź powrotna skierowana jest w $C$ (zamiast w $B$). Efekt tej zmiany to umiejscowienie największego akceptującego poprzednika poza cyklem. Mimo że wewnątrz cyklu są stany akceptujące, to nie zostają one spropagowane, bo wartość $B$ jest większa ($ B > C \land B > D $). Taki cykl zostanie pominięty przez zastosowaną heurystykę (1. podstawowy powód).
\item Maksymalny akceptujący poprzednik znajduje się w cyklu, jednak to nie on go zaczyna. Próba propagacji wartości $C$ do $B$ zakończy się fiaskiem, ponieważ $ C < B $. Dalej $B$ przekazane zostanie do $D$. W wyniku tego porównuje się ze sobą obecną wartość $D$ ze stanem, do którego wraca krawędź. $ B \neq C $, więc do wykrycia cyklu nie dojdzie.
\end{enumerate}


\def \noderadius {1.2cm}
\def \noderadiuspt {0.6}
\def \radius {1.5}
\def \marginangle {30}
\begin{figure}[h]
\centering
% w komentarzach wartość MAP oryginalna i po algorytmie

% uda się, wracamy do akceptującego
\begin{tikzpicture}
\node at (-4,1) {$ (a) $};
\draw (-4,0) node[draw, circle, minimum size=\noderadius] {$ A $}; % -/-
\draw[->, >=latex] (-4+\noderadiuspt,0) -- (-1-\noderadiuspt,0);
\draw (-1,0) node[draw, circle, minimum size=\noderadius] {$ B $}; % 5/5
\draw (-1,0) node[draw, circle, minimum size=\noderadius-0.2cm] {};
\draw[->, >=latex] (-1+\noderadiuspt,0) -- (2-\noderadiuspt,0);
\draw (2,0) node[draw, circle, minimum size=\noderadius] {$ C $}; % -/5
\draw[->, >=latex] (2+\noderadiuspt,0) -- (5-\noderadiuspt,0);
\draw (5,0) node[draw, circle, minimum size=\noderadius] {$ D $}; % -/5
\draw[->, >=latex] (5+\noderadiuspt,0) -- (8-\noderadiuspt,0);
\draw (8,0) node[draw, circle, minimum size=\noderadius] {$ E $}; % -/5
\draw[->, >=latex] (5,\noderadius/2) arc ({\marginangle}:{180 - \marginangle}:3.4);
\end{tikzpicture}

% uda się, wartość MAP ze stanu 2 przeszła na 3 i 4
\begin{tikzpicture}
\node at (-4,1) {$ (b) $};
\draw (-4,0) node[draw, circle, minimum size=\noderadius] {$ A $}; % -/-
\draw[->, >=latex] (-4+\noderadiuspt,0) -- (-1-\noderadiuspt,0);
\draw (-1,0) node[draw, circle, minimum size=\noderadius] {$ B $}; % 5/5
\draw (-1,0) node[draw, circle, minimum size=\noderadius-0.2cm] {};
\draw[->, >=latex] (-1+\noderadiuspt,0) -- (2-\noderadiuspt,0);
\draw (2,0) node[draw, circle, minimum size=\noderadius] {$ C $}; % 4/5
\draw (2,0) node[draw, circle, minimum size=\noderadius-0.2cm] {};
\draw[->, >=latex] (2+\noderadiuspt,0) -- (5-\noderadiuspt,0);
\draw (5,0) node[draw, circle, minimum size=\noderadius] {$ D $}; % 3/5
\draw (5,0) node[draw, circle, minimum size=\noderadius-0.2cm] {};
\draw[->, >=latex] (5+\noderadiuspt,0) -- (8-\noderadiuspt,0);
\draw (8,0) node[draw, circle, minimum size=\noderadius] {$ E $}; % -/5
\draw[->, >=latex] (5,\noderadius/2) arc ({\marginangle}:{180 - \marginangle}:3.4);
\end{tikzpicture}

% nie uda się -> mimo poprawnej pętli, stan 3 nie jest źródłem wartości propagowanej dalej (a)
\begin{tikzpicture}
\node at (-4,1) {$ (c) $};
\draw (-4,0) node[draw, circle, minimum size=\noderadius] {$ A $}; % -/-
\draw[->, >=latex] (-4+\noderadiuspt,0) -- (-1-\noderadiuspt,0);
\draw (-1,0) node[draw, circle, minimum size=\noderadius] {$ B $}; % 5/5
\draw (-1,0) node[draw, circle, minimum size=\noderadius-0.2cm] {};
\draw[->, >=latex] (-1+\noderadiuspt,0) -- (2-\noderadiuspt,0);
\draw (2,0) node[draw, circle, minimum size=\noderadius] {$ C $}; % 4/5
\draw (2,0) node[draw, circle, minimum size=\noderadius-0.2cm] {};
\draw[->, >=latex] (2+\noderadiuspt,0) -- (5-\noderadiuspt,0);
\draw (5,0) node[draw, circle, minimum size=\noderadius] {$ D $}; % 3/5
\draw (5,0) node[draw, circle, minimum size=\noderadius-0.2cm] {};
\draw[->, >=latex] (5+\noderadiuspt,0) -- (8-\noderadiuspt,0);
\draw (8,0) node[draw, circle, minimum size=\noderadius] {$ E $}; % -/5
\draw[->, >=latex] (5,\noderadius/2) arc ({\marginangle}:{180 - \marginangle}:1.7);
\end{tikzpicture}

% nie uda się -> kolejne stany akceptujące mają większą wartość MAP, przez co poprzednia zostaje zapomniana (c,b?)
\begin{tikzpicture}
\node at (-4,1) {$ (d) $};
\draw (-4,0) node[draw, circle, minimum size=\noderadius] {$ A $}; % -/-
\draw[->, >=latex] (-4+\noderadiuspt,0) -- (-1-\noderadiuspt,0);
\draw (-1,0) node[draw, circle, minimum size=\noderadius] {$ C $}; % 5/5
\draw (-1,0) node[draw, circle, minimum size=\noderadius-0.2cm] {};
\draw[->, >=latex] (-1+\noderadiuspt,0) -- (2-\noderadiuspt,0);
\draw (2,0) node[draw, circle, minimum size=\noderadius] {$ B $}; % 6/6
\draw (2,0) node[draw, circle, minimum size=\noderadius-0.2cm] {};
\draw[->, >=latex] (2+\noderadiuspt,0) -- (5-\noderadiuspt,0);
\draw (5,0) node[draw, circle, minimum size=\noderadius] {$ D $}; % 3/6
\draw (5,0) node[draw, circle, minimum size=\noderadius-0.2cm] {};
\draw[->, >=latex] (5+\noderadiuspt,0) -- (8-\noderadiuspt,0);
\draw (8,0) node[draw, circle, minimum size=\noderadius] {$ E $}; % -/6
\draw[->, >=latex] (5,\noderadius/2) arc ({\marginangle}:{180 - \marginangle}:3.4);
\node at (-2.7,-1.5) {$ A > B > C > D > E $};
\end{tikzpicture}

\caption{Przykłady działania heurystyki algorytmu. W przypadkach (a) i (b) akceptujący cykl zostanie wykryty, jednak dla (c) i (d) już nie.}
\label{fig:alg_heuristic_examples}
\end{figure}


\newpage
\section{Działanie w locie}

Zastosowany algorytm działa w locie (poziom 1).
W tym przypadku oznacza to, że może on zakończyć się przed eksploracją całego grafu stanów.
Stany są generowane i na bieżąco (każdy po kolei) weryfikowane.
Powodem, że to poziom 1 (a nie 2), przez który nie dzieje się tak zawsze, jest zastosowana heurystyka.
Pozwala ona wykryć część cykli akceptujących w locie, jednak są przypadki, kiedy nie wystarcza.
Wtedy cykl zostanie pominięty.
Do tego momentu algorytm ten okazuje się niewystarczający i może zwrócić niepoprawny wynik.
Zostało to uniknięte poprzez dodanie drugiej części, która odnajdzie już każdy cykl akceptujący (OWCTY).
Niestety nie działa ona w locie, więc potrzebuje wygenerowanej całej przestrzeni stanów.

\begin{figure}[h]
    \centering
    \includegraphics[height=12.8cm,keepaspectratio]{img/on-the-fly-diagram.png}
    \caption{Diagram prezentujący, kiedy dochodzi do wczesnego zakończenia algorytmu.}
    \label{fig:on_the_fly_diagram}
\end{figure}

Ustalenie, kiedy algorytm zadziała w locie, przedstawia rys. \ref{fig:on_the_fly_diagram}.
Kluczowe pytanie to czy graf zawiera wykrywalny przez heurystykę cykl akceptujący.
Jeśli tak, wtedy cała procedura zakończy się wcześnie (bez eksploracji wszystkich wierzchołków, a zaraz po domknięciu się cyklu).

W przypadku grafu posiadającego cykl akceptujący, który zostanie pominięty (zgodnie z dwoma powodami opisanymi w sekcji \ref{heuristic}), dojdzie do pełnego przeszukania, aby go wykryć (brak działania w locie).
Niespełnienie formuły zawsze implikuje przegląd wszystkich wierzchołków grafu.
Nie można orzekać niespełnienia przed sprawdzeniem całości.

\chapter{Prezentacja wyników}

\chapter{Podsumowanie}

Celem pracy było stworzenie rozproszonego algorytmu weryfikacji modelowej na podstawie własności logicznych czasu liniowego LTL dla języka Alvis z wykorzystaniem frameworka Spring.
Wykonany został system, na który składa się kilka aplikacji komunikujących się za pomocą sieci.

Rozwiązanie to pozwala na weryfikację modelową w skalowalny sposób.
Wszystko dzięki bezstanowym serwisom oraz bazie NoSQL.
Wdrożenie na platformie chmurowej umożliwiłoby sprawdzanie wielu formuł LTL jednocześnie.
Wydajność byłaby zależna od liczby uruchomionych instancji.
Ponadto algorytm działa równolegle dla grup stanów wychodzących z jednego węzła, aby wykorzystać potencjał wielu rdzeni procesora i przyspieszyć długość weryfikacji pojedynczej formuły.

Sposób implementacji pozwala też na krokowe weryfikowanie.
To aplikacja eksplorująca stany nadaje tempo i kontroluje całość.
Umożliwia to nawet pauzowanie procesu, a dzięki trwałemu zapisowi w bazie danych -- kontynuację w dowolnym momencie.

Kolejną cechą jest działanie w locie.
Pozwala to na wczesne zakończenie algorytmu, czyli zanim przeszukana zostanie cała przestrzeń stanów.
Istnieją pewne warunki, które należy spełnić, aby do tego doszło.
Jednak sam ten fakt daje możliwość kontroli zmian w modelu z dużym prawdopodobieństwem szybkiego wykrycia problemów.
Równoległe uruchomienie wielu formuł pozwoli odnaleźć w locie znaczną część nowych błędów.

Aby mieć pewność, że system spełnia wszystkie założenia, trzeba oczywiście dokonać sprawdzenia wszystkich jego stanów.
Nie można orzec spełnienia formuły, jeśli proces zakończy się wcześniej.
Zwykle wymaga to długotrwałych obliczeń, ale dzięki implementacji, która umożliwia wdrożenie w środowisku chmurowym (z cechami opisanymi powyżej), czas ten ulega skróceniu.


\section{Propozycje rozwoju}

Podobnie jak w większości systemów tutaj także jest miejsce na usprawnienia.
Odnaleźć można kilka obszarów, które można rozwinąć.

Jeden z nich to sposób komunikacji pomiędzy aplikacjami GB i LV.
Interfejs aplikacji weryfikującej opiera się na protokole HTTP z wykorzystaniem formatu JSON.
Do wydajniejszego zakodowania danych użyć można np. Protocol Buffers, FlatBuffers, czy Apache Thrift.
Co więcej -- zmiana modelu integracji na przesyłanie komunikatów pozwoliłaby na łatwiejsze zrównoleglenie pracy klienta i serwera.

Kolejny punkt na usprawnienie to baza danych.
Obecnie dwie tabele (przechowujące stany i tranzycje) używane są przez dwie aplikacje.
Jedynie GB dokonuje w nich modyfikacji, jednak potrzebne dane mogłyby być transportowane razem z komunikatem.
Pozbycie się integracji dwóch aplikacji za pośrednictwem bazy danych zmniejszyłoby wzajemne zależności w systemie.

Ostatnim dostrzeżonym miejscem do ewentualnych zmian jest tekstowa reprezentacja formuły.
Wykorzystany język Kotlin świetnie sobie radzi z tym zadaniem, zapewnia pełną elastyczność, ale stwarza też zagrożenie bezpieczeństwa.
Można za jego pomocą wykonać kod na serwerze LV, więc zastosowanie własnego języka opisu wraz z parserem byłoby bezpieczniejsze (lecz zapewne bardziej ograniczające).


\printbibliography

\end{document}
