\chapter{Architektura systemu}

Założeniem projektu było stworzenie algorytmu weryfikacji modelowej w oparciu o własności LTL dla języka Alvis w środowisku rozproszonym.


\section{System rozproszony}

Architektura systemu została zaplanowana tak, aby wydzielić funkcjonalności do odseparowanych aplikacji.
Całość składa się z 3 programów oraz bazy danych.
Jej schemat ogólny został przedstawiony na rys. \ref{fig:system_overview}.

\begin{figure}[h]
    \centering
    \includegraphics[width=\linewidth,keepaspectratio]{img/system_overview.png}
    \caption{Schemat ogólny systemu.}
    \label{fig:system_overview}
\end{figure}

Pierwszym elementem jest \textit{Graph Provider} (\textbf{GP}).
Spełnia on dwa zadania.
Pierwsze z nich to dostarczanie wszystkich osiągalnych tranzycji dla danego stanu.
Drugie umożliwia otrzymanie stanów dla zadanej tranzycji, które można odwiedzić z wybranego stanu źródłowego.
Serwis ten enkapsuluje działanie całej domeny.
Jako jedyny dostarcza dane, dzięki którym da się zbudować przestrzeń stanów.
GP spełnia prosty interfejs przedstawiony w listingu \ref{lst:graph_provider_interface}.

\begin{minipage}{\linewidth}
\begin{lstlisting}[caption={Interfejs implementowany przez GP.},captionpos=b,label={lst:graph_provider_interface}]

  public interface GraphProvider {
    Collection<Transition> allTransitionFromState(SystemState from);

    Collection<SystemState> allReachableStates(SystemState from,
                                               Transition through);
  }
\end{lstlisting}
\end{minipage}

\textit{LTL Verifier} (\textbf{LV}) to serwerowa aplikacja zajmująca się weryfikacją modelową i zawiera główną część tego algorytmu.
Konwertuje ona także formuły LTL do automatów Büchiego.
Napisana została w języku Java z wykorzystaniem frameworka Spring.
Interfejs implementowany przez LV opisuje listing \ref{lst:ltl_verifier_interface}.

\begin{lstlisting}[caption={Interfejs implementowany przez LV.},captionpos=b,label={lst:ltl_verifier_interface}]

  public interface LtlVerifier {
    VerificationId initializeVerification(VerificationInit verificationInit);

    VerificationResult newStates(NewStates newStates,
                                 VerificationId id);

    VerificationResult finish(VerificationId id);
  }
\end{lstlisting}

\textit{Graph Builder} (\textbf{GB}) odgrywa rolę serca systemu.
To centrum sterowania, które inicjuje zapytania do pozostałych aplikacji.
Zarządza eksploracją kolejnych stanów, a także wysyła je do weryfikacji.
Zarówno GB, jak i LV komunikują się z bazą danych (Cassandra), aby czytać i zapisywać stany czy tranzycje.
Komunikacja między GB a GP zachodzi za pomocą Apache Thrift, wykorzystując binarny protokół do  przesyłania danych siecią.
Zapytania GB do LV realizowane są poprzez protokół HTTP w formacie JSON.


\section{Alvis}

Alvis to język formalny, którego celem jest dostarczenie elastyczności w modelowaniu systemów współbieżnych i czasu rzeczywistego wraz z możliwością weryfikacji opartej o metody formalne.
Jego nazwa pochodzi od połączenia słów \textbf{al}gebra i \textbf{vis}ual.
To łatwy w użyciu dla inżynierów formalny język modelowania.
Oferuje także język wizualny do projektowania struktur.
Za pomocą Alvisa modelować można systemy współbieżne, czasu rzeczywistego, wbudowane oraz rozproszone.

Model Alvisa to system agentów, które zazwyczaj działają współbieżnie, komunikują się między sobą, konkurują o współdzielone zasoby itp.
Agenty dzielą się na aktywne i pasywne.
Zachowanie każdego z nich definiuje się w warstwie kodu, którego konstrukcje przypominają języki wysokiego poziomu.

W tworzonym rozwiązaniu potrzebny był model systemu, dla którego przeprowadzona zostanie weryfikacja.
Wykorzystany został taki, który modeluje algorytm eksploracji stanów dla języka Alvis (czyli aplikacje GB, GP oraz bazę danych).
To konstrukcja, która pozwala sprawdzić samą siebie, a więc czy algorytm budujący pełny graf stanów nie zawiera błędów.
W jej skład wchodzi wiele elementów, np. stan heurystyki przeszukiwania BFS, bazy danych, wątku budującego graf, czy repozytoriów na dane.
To one będą weryfikowane pod kątem spełnienia formuł LTL.
